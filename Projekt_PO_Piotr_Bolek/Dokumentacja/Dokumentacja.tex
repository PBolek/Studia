\documentclass[a4paper]{article}
\usepackage[polish]{babel}
\usepackage[T1]{fontenc}
\usepackage[utf8]{inputenc}
\usepackage{graphicx}
\usepackage{enumerate}
\frenchspacing

\title{Projekt - Programowanie obiektowe 2018}
\author{Piotr Bolek}
\date{\today}

\begin{document}
	
	\begin{enumerate}
		\huge \item \textbf{Opis projektu} \\
			\normalsize 
				Celem projektu było napisanie gry planszowej dla wielu graczy z podziałem na tury. Gra polega na przemieszczaniu się postacią po planszy z 
				polami, tak jak w "Chińczyku" i zebraniu jak największej ilości kryształów. Gracz może poruszać się w górę, w dół, w lewo i w prawo. Każdy
				gracz posiada trzy przedmioty którymi może wpływać na wygląd planszy, czy też na stan rozgrywki. Pierwszym z nich jest kilof. Za pomocą 
				tego przedmiotu gracz może tworzyć nowe ścieżki czy też niszczyć pozostawione przez innych graczy ściany. Również może usuwać czarne, 
				niebieskie i żółte pola. Drugim przedmiotem jest łopata, za pomocą której każdy gracz może zebrać kryształy ukryte pod niebieskimi polami, 
				które w każdej rozgrywce są ustawiane losowo na planszy. Trzecim i ostatnim przedmiotem jest bomba, przy pomocy której gracz może 
				ustawić ścianę. Gra jest podzielona na tury. Kolejność graczy jest, z każdym uruchomieniem gry, generowana losowo, przez co nie ma 
				znaczenia kto jaki number wyberze, a gra i tak będzie za każdym razem uczciwa. Wygrywa drużyna, która uzbiera najwięcej kryształów 
				do momentu zakończenia gry.
			\huge \item \textbf{System gry}
				\begin{enumerate}
					\Large \item \textbf {Pola}
						\begin{enumerate}
							\large \item \textbf {Zwyczajne} \\
								\normalsize
									Do pól zwyczajnych zaliczamy: zielone oraz brązowe pola. Wejście przez gracza na pole zwyczajnie nie wpływa w żaden
									sposób na rozgrywkę czy też na stan gracza. Po użyciu kilofa powstaje zielone pole. \\
							\large \item \textbf {Pola specjalne} \\
								\normalsize
									Do pól specjalnych zaliczamy: czarne, niebieskie, fioletowe oraz żółte pola. \\
									\begin{enumerate}
										\large \item \textbf {Pole czarne} \\
											\normalsize 
												Po wkroczeniu przez gracza na pole czarne, zostaje wywołane negatywne losowe zdarzenie.
												Losowaniu podlegają cztery zdarzenia w przypadku normalnego pola czarnego: utrata jednej
												łopaty, utrata jednej bomby, utrata jednego kilofa lub stracenia losowej ilości (od 1 do 9)
												punktów ruchu gracza, który wszedł na czarne pole.\\
										\large \item \textbf {Pole niebieskie} \\
											\normalsize
												Po wkroczeniu przez gracza na niebieskie pole nic się nie dzieje. Dopiero gry gracz użyje
												łopaty, to wykopuje jeden, losowy, kryształ. Na pierwszym poziomie planszy (zielone pola)
												jest mniejsze prawdopodobieństwo otrzymania lepszego kryształu niż na poziomie drugim
												(brązowe pola).\\
										\large \item \textbf {Pole żółte} \\
											\normalsize
												Pola żółte po wkroczeniu na nie przez gracza znikają. Efektem wejścia na pole żółte jest jedno
												z następujących wydarzeń: zwiększenie liczby łopat o 1, zwiększenie liczby kilofów o 1, zwiększenie
												liczby bomb o 1, zwiększenie punktów ruchu (modulo 5). \\
										\large \item \textbf {Pola fioletowe} \\
											\normalsize
												Pola filoetowe po wkroczeniu na nie przez gracza przenoszą go na inne, losowe, pole fioletowe na mapie.
												Po wejściu na pole filoetowe i przeniesieniu gracza w inne miejsce na mapie, gracz posiada jeden punkt
												ruchu niezależnie od poprzeniej ilości punktów ruchu. \\									
									\end{enumerate}
						\end{enumerate}
					\Large \item \textbf {Przedmioty} \\
						\normalsize
							Każdy gracz posiada trzy przedmioty do użycia. Wpłwają one na wygląd planszy jak i na przebieg rozgrywki. Standardową ilość wszystkich
							trzech przedmiotów to 5. Gracz może uzupełnić stan swoich przedmiotów do 5, gdy powróci do miasta (miejsce w którym zaczyna każdy gracz).
							\begin{enumerate}
								\large \item \textbf {Kilof} \\
									\normalsize
										Za pomocą tego przedmiotu gracz może tworzyć nowe ścieżki. \\
								\large \item \textbf {Łopata} \\
									\normalsize											
										Za pomocą tego przedmiotu gracz może podnosić klejnoty ukryte pod niebieskimi polami. \\
								\large \item \textbf {Bomba} \\
									\normalsize
										Za pomocą tego przedmiotu gracz może tworzyć ściany, przez które nikt nie może przejść. Jedyną możliwością usunięcia ściany jest wykopanie
										jej przy pomocy kilofa. \\
							\end{enumerate}
					\Large \item \textbf {Tury} \\
						\normalsize
							Kolejność graczy jest losowa. Ilość punktów ruchu gracza w danej turze jest losowa (modulo 11).
				\end{enumerate}
		\huge \item \textbf {Opis klas} \\
			\Large \item \textbf {Application} \\
			\Large \item \textbf {Board} \\
			\Large \item \textbf {Map} \\
			\Large \item \textbf {Menu} \\
			\Large \item \textbf {End game} \\
			\Large \item \textbf {Turn} \\
			\Large \item \textbf {Random events} \\
			\Large \item \textbf {Equipement} \\
			\Large \item \textbf {Player} \\
			\Large \item \textbf {Shovel, Pickaxe, Bomb} \\
			\Large \item \textbf {Yellow jewel, Green jewel, Blue jewel, White jewel} \\
			\Large \item \textbf {Game stats} \\
	\end{enumerate}
\end{document}